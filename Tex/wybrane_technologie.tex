\chapter{Wybrane technologie}

W tym rozdziale opisane zostaną technologie użyte do rozwiązania zadanego problemu.
Aplikacja została podzielona na 3 warstwy wymagające zastosowania różnych rozwiązań,
i zostaną one omówione w swoich sekcjach.

\section{Aplikacja kliencka}
Aplikacja kliencka została wykonana w formie strony internetowej.

\subsection{WebAssembly}
WebAssembly to technologia opisująca standardowy kod binarny oraz jego reprezentację tekstową,
niezależne od platformy jego wykonania \cite{mdn:wasm, wasm:standard}. Umożliwia to
tworzenie przenośnego oprogramowania, które może zostać wykorzystane w wielu różnych systemach.
Głównymi celem WebAssembly jest umożliwienie tworzenia szybkiego oraz bezpiecznego pod względem
obsługi pamięci oprogramowania, które może być stworzone w dowolnym języku programowania, 
niezależnie od platformy oraz sprzętu, na którym ma być wykonywane \cite{wasm:standard}.

\subsection{Rust}
Rust to wielo-paradygmatowy kompilowany język programowania. Oznacza to, że łączy ze sobą cechy różnych
konwencje wielu sposobów tworzenia oprogramowanie m.in. programowanie obiektowe, funkcjonalne
czy strukturalne i umożliwia osobie je tworzącej na korzystanie z dowolnych zasobów oraz
schematów danego paradygmatu. 

Celem tego języka jest umożliwienie tworzenia wydajnego oprogramowania jednocześnie
dbając o bezpieczeństwo pamięci oraz konkurencji \cite{infoworld:what_is_rust}. 
Uzyskane jest to poprzez wykorzystanie mechanizmu sprawdzania zapożyczeń (ang. \textit{borrow checker}).
Mechanizm ten sprawdza czas życia danego obiektu podczas kompilacji co sprawia, że nie wymagane
jest to podczas działania oprogramowania w przeciwieństwie do systemów z klasycznym zliczaniem referencji.
Dzięki temu nie jest wymagane użycie algorytmów odśmiecania pamięci
(ang. \textit{garbage collection}) lub zliczania referencji (ang. \textit{reference counting}), 
które dodają dodatkowe koszty wykonywania obliczeń.

Środowisko programistyczne Rust zapewnia również wiele przydatnych narzędzi
ułatwiających tworzenie oprogramowania \cite{klabnik:rust}.
Jednym z nich jest serwer języka rust-analyzer, który może zostać zintegrowany
z wieloma popularnymi środowiskami deweloperskimi oraz edytorami tekstu,
w celu zapewnienia dodatkowych funkcjonalności takie jak
autouzupełnianie czy wyświetlanie błędów kompilacji podczas edycji kodu.
Kolejnym z nich jest oprogramowanie Cargo spełniające dwie główne funkcje - 
jest systemem budującym projekt oraz umożliwia na zarządzanie zależnościami.
Pozwala to na łatwe umieszczanie kodu innych publicznie dostępnych projektów
w oprogramowaniu docelowym oraz wykorzystywanie jego funkcjonalności.

Cechy te sprawiają, że ten język programowania regularnie cieszy się z
wysokiego zadowolenia użytkowników w rankingach StackOverflow.
W roku 2022 9,32\% respondentów używało języka Rust i aż 86,73\% z nich
uznało, że kocha z nim pracować \cite{stackoverflow:popularity}.
Jest to najwyższy wynik z jakiejkolwiek technologii w tej ankiecie,
drugie miejsce zajmuje język Elixir z wynikiem 75,46\%.
Dzięki tak dużej popularności powstało wiele otwarto-źródłowych bibliotek,
które mogą pomóc w rozwiązaniu zadanego problemu.

\subsection{Yew}
Yew to framework umożliwiający tworzenie witryn WWW w języku Rust.
Z użyciem tej biblioteki możliwe jest konstruowanie niezawodnych
oraz wydajnych stron internetowych. Yew wykorzystuje WebAssembly wraz
z renderowaniem po stronie serwera, aby znacznie przyśpieszyć działanie
aplikacji na docelowej platformie.

Do tworzenia stron Yew wykorzystuje język podobny do HTML i JSX dzięki
czemu jest przystępny dla programistów mających wcześniejsze doświadczenie
z tymi technologiami. Podobnie jak inne środowiska takie jak React czy Angular
tworzone są komponenty, które mogą być wielokrotnie używane w różnych miejsach
na stronie co sprawia, że kod jest czytelniejszy i jest jego mniej.

\subsection{Bootstrap}
Bootstrap to darmowe i otwarto-źródłowe środowisko programistyczne składające się
ze zbioru szablonów stworzonych w językach HTML, CSS i JavaScript, które mogą
być dowolnie wykorzystane z kompatybilnymi technologiami.
Ten framework zawiera podstawowe style potrzebne, aby stworzyć responsywną 
aplikacje webowe na wielu urządzeniach z różnymi rozmiarami ekranów oraz rozdzielczością. 

\section{Aplikacja serwerowa}

\subsection{.NET Core}
.NET Core to następca popularnej platformy programistycznej .NET Framework stworzonej przez
firmę Microsoft. W przeciwieństwie do poprzednika, którego oficjalne wsparcie ograniczone było
do systemu Windows, a same oprogramowanie było własnościowe, jest to technologia otwarto-źródłowa
wspierająca wiele systemów operacyjnych \cite{price2021c}. Możliwe jest to poprzez zastosowanie
języka pośredniego niezależnego od systemu lub sprzętu. Instrukcje te tłumaczone są w 
wirtualnej maszynie CLR (ang. \textit{Common Language Runtime}) na język maszynowy, a
następnie wykonywane jak standardowy natywny kod \cite{msdn:clr}. Dzięki takiemu podjeściu
możliwe jest wytworzenie przenośnego oprogramowania, które może być wykorzystane na wielu
systemach.

\subsection{C\#}
C\# to silnie typowany język obiektowy wysokiego poziomu stworzony przez firmę Microsoft.
Do kompilacji wykorzystywane jest środowisko .NET, dzięki czemu zachowane są
jego cechy takie jak przenośność oprogramowania oraz możliwość wykorzystania
innych języków programowania bazujących na tej technologii.

Dzięki automatycznemu zarządzaniu pamięcią ułatwia znacznie tworzenie
oprogramowania, gdyż w bezpiecznym kontekście nie pozwala na 
błędne jej wykorzystanie. Wykonywane jest to poprzez zastosowanie
mechanizmu odśmiecania pamięci (ang. \textit{garbage collection}).

C\# zawiera również rozszerzenia kolekcji LINQ (ang. \textit{Language Integrated Query})
pozwalające na wykonywanie na nich dodatkowych operacji, tj. grupowanie, sortowanie czy wyszukiwanie.
Te dodatkowe metody mogą być wykorzystane do działań na róznych źródłach danych takich jak dokumenty XML
czy też bazy danych SQL \cite{zhang2014}. Ich wysokopoziomowy kod jest przenośny i może być
wykorzystywany w takiej samej formie przy różnych źródłach danych.

Kolejną zaletą tej technologii jest możliwość wykorzystania meta-programowania. 
Jest to technika umożliwiająca modyfikację kodu podczas jego kompilacji lub wykonywania. 
Dzięki niej podczas wykonywania programu, poprzez refleksję, możliwe jest rozpoznanie
typu obiektu oraz jego modyfikacja. Pozwala również na tworzenie drzew wyrażeń, które
mogą zostać dynamicznie skompilowane w celu, np. wykorzystania zaawansowanej logiki
biznesowej, która może być zmienna.

\subsection{ASP.NET}
ASP.NET to modularna platforma programistyczna oparta na technologii .NET. 
Jej bogata funkcjonalność pozwala na tworzenie stron internetowych, API HTTP oraz 
wiele innych aplikacji webowych poprzez zastosowanie wielu z jej rozszerzeń.

\subsection{MongoDB}
MongoDB to wieloplatformowy system bazodanowy. Dane przechowywane są w postaci dokumentów
w formacie podobnym do JSON. Umożliwia dzięki temu, w porównaniu do klasycznych baz
danych SQL, na większą elastyczność dzięki braku konieczności określania schematu tabel.

\section{Urządzenie pomiarowe}

\subsection{Raspberry Pi Pico W}

\subsection{C++}
C++ to obiektowy język programowania wysokiego poziomu, często nazywany również "C z klasami".
Obsługuje on wysokopoziomowe abstrakcje takie jak obiekty, programowanie generyczne czy
abstrakcje pozwalając w tym samym czasie na używanie niskopoziomowej funkcjonalności 
takiej jak zarządzenie pamięcią czy używanie języka maszynowego. Przez te cechy
używany jest tam, gdzie potrzebna jest wysoka wydajność lub zasoby komputerowe
są znacznie ograniczone, np. serwery WWW, gry wideo czy systemy wbudowane.

W porównaniu do C zawiera również rozbudowaną standardową bibliotekę klas i funkcji.
Dodaje ona dodatkową funkcjonalność taką jak zaawansowane typy pozwalające na
przechowywanie zmiennej liczby danych, algorytmy takie jak sortowanie czy 
generowanie liczb losowych, obsługa plików, czasu oraz wielowątkowości.

\subsection{Arduino}


