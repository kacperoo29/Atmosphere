\chapter{Analiza istniejących rozwiązań}
Ze względu na powszechność problemu przedstawionego w tej pracy na przestrzeni lat powstało wiele 
systemów umożliwiających rozwiązanie go w mniejszym lub większym stopniu. 
Ich eksploracja może ułatwić wytworzenie podobnego systemu poprzez zauważenie mocnych i słabych
stron danego rozwiązania. Na ich podstawie możliwa jest adopcja oraz usprawnienie danych składowych
systemu w przypadku zalet oraz poprawa lub zastąpienie innym rozwiązaniem w przypadku wad.


W tym rozdziale zostaną poddane analizie niektóre z tych systemów w celu określenia ich 
wad i zalet oraz wyszczególnienia brakującej, bądź też niepełnej funkcjonalności.

\section{Kryteria analizy porównawczej}
Analiza porównawcza odbędzie się na podstawie następujących kryteriów:
\begin{itemize}
  \item typ parametrów, które dany system pozwala obserwować;
  \item zakres pomiarów;
  \item częstotliwość pomiarów;
  \item sposób komunikacji;
  \item sposób przechowywania danych;
  \item koszt wdrożenia;
  \item możliwość rozbudowy.
\end{itemize}
Zastosowanie tych parametrów powinno umożliwić szczegółowy przegląd dostępnych aktualnie
rozwiązań oraz wskazanie punktów, które mogą zostać w znaczny sposób usprawnione.

\section{System Verkada SV11}
Verkada SV11 jest wielofunkcyjnym urządzeniem \cite{verkada:sv11} pozwalającym na odczyt wielu
parametrów środowiska. System ten jest w stanie odczytać poniższe wartości:
\begin{itemize}
  \item temperaturę pomieszczenia w zakresie $-5 - 50 ^\circ C$,
  \item wilgotność powietrza w zakresie $0-80\%$,
  \item aerozole atmosferyczne o średnicy mniejszej niż $2,5\mu m$ (PM2,5) w zakresie $0-1000\mu g/m^3$,
  \item poziom głośności w zakresie $20 - 120dB$,
  \item wskaźnik jakości powietrza (ang. \emph{Air Quality Index}) zgodnie ze standardem USEPA (ang. \emph{United States Environmental Protection Agency}),
  \item ruch za pomocą detektora podczerwieni.
\end{itemize}
Pomiary wykonywane są w czasie zbliżonym do czasu rzeczywistego wykorzystując sieć Internet
poprzez połączenie typu Ethernet z zasilaniem w technologii PoE (ang \emph{Power over Ethernet}). 
Technologia wykorzystywana w sensorach firmy Verkada wymaga stałego połączenia z chmurą, 
dzięki czemu użytkownik nie musi dbać o przechowywanie czy przetwarzanie danych, ale
wiąże się to z dodatkowymi opłatami abonamentowymi i limitowaną funkcjonalnością w zależności
od poziomu subskrypcji. System pozwala na przeglądanie danych aktualnych oraz historycznych, a
przy dodatkowej opłacie udostępnia tryb alertów pozwalający na notyfikowanie użytkowników o
niespodziewanych zdarzeniach, takich jak przekroczenie wartości danych parametrów.
Koszt systemu jest najwyższy z przedstawionych w tej analizie i dodatkowo bazuje on na
modelu subskrypcyjnym co sprawia, że może być on poza zasięgiem dla większości osób i jest
głównie przeznaczony dla dużych firm.
Pod warunkiem wykorzystania jedynie urządzeń tej firmy system może być rozbudowany o 
dodatkowe sensory, kamery, alarmy czy też urządzenia kontroli dostępu. Urządzenia oraz
oprogramowanie innych producentów nie może z nimi współpracować.

\section{System SensorPush}
System SensorPush umożliwia na monitorowanie podstawowych parametrów takich jak temperatura,
wilgotność i ciśnienie powietrza. Produkt jest w stanie odczytywać z dużą dokładnością pomiary temperatury
w zakresie $0 - 60 ^\circ C$, wilgotność powietrza w zakresie $0-80\%$ i ciśnienie powietrza w zakresie 
$300 - 1250mb$. Urządzenie pracuje całkowicie używając zasilanie bateriami CR2477 co zwiększa
jego przenośność, gdyż nie potrzebne jest zewnętrzne źródło energii elektrycznej. 
W wersji bez dodatkowej bramy WiFi produkt wykorzystuje połączenie z użyciem technologii
Bluetooth w celu przesyłania odczytów do aplikacji sterującej. 
Najmniejszy możliwy interwał odczytów wynosi 1 minutę.
System pozwala na połączenie wielu sensorów tego producenta z jego
oprogramowaniem i monitorowanie aktualnych oraz historycznych odczytów. Główną limitacją
jest połączenie Bluetooth i jego zasięg przez co nie zawsze jest możliwe uzyskanie dobrego
zasięgu obsługi. Producent oferuje dodatkową bramę WiFi, z którą możliwe jest sparowanie
sensorów oraz następnie przesyłanie przez nią danych i zapis w chmurze. Ta usługa płatna
jest tylko przy zakupie urządzenia następnie producent zapewnia do niej dostęp bez dodatkowych opłat.
Dzięki temu rozwiązaniu możliwy jest dostęp do danych wszędzie z dostępem do sieci Internet.
Dodatkowo aplikacja wyposażona jest w system alertów, które użytkownik może otrzymywać przy odpowiedniej
konfiguracji, np. przy przekroczeniu danej wartości temperatury. Te notyfikacje prezentowane są
jako powiadomienia wykorzystując urządzenie, na którym zainstalowana jest aplikacja.

\section{System TempStick}
TempStick jest prostym urządzeniem pozwalającym na monitorowanie temperatury oraz wilgotności powietrza.
Urządzenie pozwala na odczyt temperatury w zakresie $5 - 60 ^\circ C$ i wilgotności w zakresie $0-100\%$.
Zasilany jest z użyciem baterii AA i nie wymaga dodatkowego źródła energii.
Przesyłanie danych odbywa się z użyciem sieci WiFi po wcześniejszym sparowaniu urządzenia z aplikacją
producenta. Odczyty z sensorów przesyłane są bezpośrednio do serwisu chmurowego, skąd mogą być następnie
pobrane w celu sprawdzenia aktualnych i historycznych wartości. Producent zapewnia do nich dostęp bez
dodatkowych opłat. System pozwala na połączenie wielu sensorów tego wytwórcy do jednego konta
użytkownika co umożliwia jednoczesny odczyt ich danych. Minimalny interwał czasu pojedynczego
odczytu wynosi 5 minut. System umożliwia również konfigurację notyfikacji przy przekroczonych 
wartościach oraz alertów o utraconym połączeniu czy wyczerpującej się baterii. Zakładając istnienie 
wcześniejsze infrastruktury WiFi koszt wprowadzenia jest niski.

\section{Podsumowanie}
Przedstawione urządzenia cechują się podobnym zakresem możliwych pomiarów. Można z tego wnioskować, że
większość sensorów dostępnych aktualnie na rynku będzie zachowywać się podobnie. Nie jest więc to
zagadnienie, które można w łatwy sposób usprawnić.


Zupełnie inaczej jest w przypadku typu pomiarów jakie możemy wykonywać. Podstawowymi parametrami są
temperatura i wilgotność powietrza, które można odczytać za pomocą większości rozwiązań tego typu.
Systemy z możliwością uchwycenia dodatkowych czynników środowiska zazwyczaj są znacznie droższe, a
urządzenia nie mogą być w przyszłości rozbudowane o dodatkowe sensory. Jest to punkt, który jest możliwy
do poprawy z użyciem systemu Raspberry Pi i wytworzeniu adekwatnie uniwersalnego oprogramowania, które umożliwiałoby
użytkownikowi na definiowanie własnych parametrów oraz dodawanie nowych sensorów.


Częstotliwość pomiarów również jest punktem, który mógłby ulec polepszeniu. Oprócz najdroższej opcji
przedstawionej w tej pracy żadna z nich nie jest w stanie działać w czasie zbliżonym do rzeczywistego
i pomiary wykonywane są w odstępach liczonych w minutach. Może być to niewystarczające w wielu przypadkach
użycia takiego systemu, a z użyciem odpowiedniej platformy interwał na poziomie kilku sekund nie powinien być
trudny do osiągnięcia.


Każda z przedstawionych solucji wykorzystuje różne sposoby połączenia urządzeń z siecią. Ze wszystkich
najlepszym wydaje się być ten z systemu SensorPush. W dzisiejszych czasach sieci WiFi są bardzo powszechne
i nie powinno to w większości przypadków generować dodatkowych kosztów. W porównaniu z wykorzystaniem 
technologii Bluetooth sieci bezprzewodowe WiFi cechują się znacznie większą niezawodnością oraz
większym zasięgiem działania. Natomiast sieć Ethernet mimo, że jest najbardziej niezawodna z nich
niesie za sobą dodatkowe problemy logistyczne takie jak rozłożenie kabli w budynkach, co w niektórych
przypadkach jak, np. obiekty zabytkowe może być niemożliwe.


Wszystkie przedstawione systemy wykorzystują technologie chmurowe do przechowywania danych.
Niesie to ze sobą wiele korzyści takich jak ułatwiony dostęp w każdym miejscu z połączeniem z
siecią Internet, ale niesie też ze sobą w niektórych przypadkach dodatkowe koszty.
Innym problemem jest przypadek braku dostępu do danego serwisu lub całkowite wygaśnięcie oferowanej
usługi. W takim przypadku dane urządzenia stają się bezużyteczne, a cały system należy migrować
do rozwiązań innego producenta. Problem ten można prosto rozwiązać oddając kontrolę nad
serwerem klientowi. Oczywiście wymaga to, aby proces wdrażania systemu był jak najprostszy, żeby
mógł trafić on do jak największej liczby osób.


Żaden z przedstawionych systemów nie pozwala na prawdziwą rozbudowę poza tym co oferuje 
dany producent. Sensory różnych producentów nie mogą być połączone z systemem innych.
Może zostać to rozwiązane poprzez publikację standardu, który taka aplikacja obsługuje.
Pozwoliłoby to na implementację różnych rozwiązań bazujących na dowolnej platformie.
