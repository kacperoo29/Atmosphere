\documentclass[12pt,a4paper]{article}
\usepackage[utf8]{inputenc}
\usepackage[polish]{babel}
\usepackage{pdfpages}
\usepackage[T1]{fontenc}
\usepackage{fontspec}
\usepackage{lipsum}
\usepackage{setspace}
\usepackage{ragged2e}
\usepackage{tocloft}
\usepackage{indentfirst}
\usepackage[hidelinks]{hyperref}

\setmainfont{Times New Roman}

\renewcommand{\cftsecleader}{\cftdotfill{.}}
\renewcommand{\cftsubsecleader}{\cftdotfill{.}}
\renewcommand{\cftsecfont}{\normalfont}

\author{Kacper Bajeński}
\date{2022}

\begin{document}
\tableofcontents
\newpage

\setstretch{1.5}
\section{Wstęp}
Badanie parametrów środowiska jest często wykonywaną czynnością w wielu
różnych dziedzinach nauki oraz w ostatnich latach znajduje swoją niszę
również w zastosowaniach komercyjnych, m.in. w systemach typu inteligentny
dom (ang. Smart Home) czy też automatyzacji urzadzeń. Mimo szerokiego
zakresu zastosowań większość wymogów, które taka aplikacja bądź zbiór
aplikacji ma spełniać jest możliwa do zrealizowania poprzez wykorzystanie
odpowiedniej architektury oraz poziomów abstrakcji. Dzięki temu możliwe
jest uzyskanie systemu, który może zostać wykorzystany w wielu zastosowaniach
oraz jego rozbudowa bądź zmiana poszczególnych składników jest znacznie ułatwiona.

Ważnym jest zaobserwowanie jakie wymagania można postawić przed takim systemem i
dobranie do tych problemów odpowiednich rozwiązań biorąc pod uwagę wydajność oraz
rozszerzalność systemu lub umożliwenie użytkownikowi na jego ulepszenie, aby dane wymogi
zostały spełnione.

W najbardziej podstawowym zastosowaniu aplikacja powinna spełniać jej główne założenia:
\begin{itemize}
      \item pomiar parametrów środowiska
      \item transmisja danych
      \item powiadomienia o przekroczeniu zadanego zakresu wartości
      \item przeglądanie aktualnych danych
      \item przeglądanie danych historycznych
\end{itemize}

Celem pracy jest wykonanie projektu takiej architektury oraz jej prototypu bazującym
na specyficznym sprzęcie jakim jest mikrokomputer Raspberry Pi. Wybierając technologie do jego
wykonania głównym czynnikiem, na który należy zwrócić uwagę jest przenośność, czyli
możliwość wykorzystania na różnym sprzęcie z jak najmniejszą ilością zmian, lub najlepiej
bez nich. Kolejnym ważnym aspektem do przemyślenia będzie wykorzystana technologia
komunikacji między urządzeniami w celu objęcia jak największej ilości platform.

\section{Wymagania}
Ten rozdział przedstawia zbiór wymagań funkcjonalnych oraz niefunkcjonalnych, które
stworzony system powinien spełniać.



\subsection{Wymagania funkcjonalne}
\subsubsection*{Pomiar parametrów}
Urządzenia pomiarowe powinny regularnie odczytywać parametry środowiska i
przesyłać je do centralnego serwera, który zajmie się ich przetworzeniem.

\subsubsection*{Pomiar na zapytanie}
Na otrzymane rządanie od serwera urządzenie pomiarowe powinno być w stanie
odczytać parametry środowiska, a następnie wysłać odpowiedź z ich szczegółami.

\subsubsection*{Wysyłanie powiadomień}
Serwer po otrzymaniu parametrów z urządzenia pomiarowego powinien je przetworzyć,
a następnie w przypadku przekroczenia zadanych ram wartości dla danego pomiaru
powiadomić o tym wydarzeniu użytkownka.

\subsubsection*{Odczyt aktualnych parametrów}
Użytkownik powinien mieć możliwość, w dowolnym momencie pracy systemu,
na odczyt aktualnych pomiarów z dowolnego urządzenia pomiarowego.

\subsubsection*{Odczyt historycznych parametrów}
Użytkownik powinien mieć dostęp do listy historycznych pomiarów.



\subsection{Wymagania niefunkcjonalne}
\subsubsection*{Rozszerzalność}
System powinien w łatwy i przejrzysty sposób pozwolić na dodawanie nowej funkcjonalności
bez potrzeby modyfikowania istniejącego już kodu. Pozwoli to na powstanie stałej, przetestowanej
bazy funkcjonalności, które będą mogły być wykorzystane w tworzeniu nowych wymagań.

\subsubsection*{Wykorzystanie darmowych rozwiązań}
Dzięki wykorzystaniu darmowych rozwiązań o odpowiedniej licencji, np. MIT czy GPL możliwe
będzie wykonanie aplikacji bez niepotrzebnych dodatkowych kosztów oraz reprodukcja pod
podobną licencją.

\subsubsection*{Przenośność}
Ostatecznie system powinien być jak najbardziej przenośny i działać bez dodatkowych zmian
na jak największej liczbie platform.

\subsubsection*{Wdrażanie}
System powinien zawierać sposób na ułatwione wdrażanie (ang. deployment) oprogramowania
na nowych platformach bądź podczas uaktualnień. Sam ten proces powinien być przeprowadzany
jak najmniejszym nakładem pracy.


\section{Wykorzystane technologie}
\subsection{Protokół HTTP i REST}
Do komunikacji został wykorzystany protokół HTTP oraz związany z nim styl architekturalny
REST.

HTTP (ang. Hypertext Transfer Protocol) jest powszechnie stosowanym protkołem służącym
do przesyłania dokumentów hipertekstowych z wykorzystaniem TCP/IP zwanym też protokołem
internetowym (ang. Internet Protocol). Dzięki jego powszechności jest kompatibilty z szeroką
gammą urządzeń i oprogramowań co w wielu przypadkach obniża koszt wykonania systemu. Jest
on stosowany w sieciach lokalnych jak i w sieci Internet do komunikacji między urządzeniami.
W związku z tym w wielu sytuacjach istnieje już infrastruktura umożliwiająca jego wykorzystanie.

\subsection{Specyfikacja API OpenAPI}
Specyfikacja API OpenAPI wykorzystywana jest do definicji zasobów i możliwości
danej usługi bez poznawania jej implementacji. Stosując się do niej możliwe jest
tworzenie oprogramowania RESTowego niezależnego od platformy i technologii wykorzystanej
do implementacji.

Odpowiednio przygotowana dokumentacja powinna zawierać wszystkie elementy usługi tj.
punkty końcowe (ang. endpoint) aplikacji oraz ich ścieżki wraz z parametrami, które
przyjmuje oraz typ odpowiedzi.

\subsection{.NET Core}
.NET Core to darmowa i otwartoźródłowa platforma programistyczna stworzona przez firmę Microsoft. Jest naturalną 
kontynuacją wcześniejszej platformy .NET Framework. Główne ich różnice to:
\begin{itemize}
      \item .NET Core jest opgoramowaniem całkowicie otwartoźródłowym, .NET Framework natomiast jest zamknięto-źródłowy.
      \item .NET Core działa na wielu systemach operacyjnych w tym Windows, Linux czy MacOS, .NET Framework jest skierowany
            jedynie do użytku w systemach Microsoft Windows.
      \item .NET Core jest stworzony całkowicie od zera z myślą o nowoczesnym sprzęcie i architekturach oprogramowania,
            dzięki czemu jest w takich zastosowaniach znacznie szybszy od poprzednika.
      \item Dzięki modularnej budowie wciąż możliwe jest wykorzystanie technologi specyficznych dla
            systemu Windows, w tym samym czasie nie dodając dodatkowego obciążenia w przypadku
            innych systemów.
\end{itemize}

Środowisko .NET standardowo obsługuje trzy języki programowania:
\begin{itemize}
      \item C\# - język obiektowy ogólnego przeznaczenia, najczęściej wykorzystywany
      \item F\# - język funkcyjny
      \item Visual Basic - następca klasycznego języka VB znanego z niskiego progu wstępu
\end{itemize}
Korzystając z dowolnego z nich możemy uzyskać te same efekty oraz stosować te
same biblioteki należące do tego środowiska. Jednak ze względu na swoje podobieństwo
do języka C oraz obiektowość C\# jest najpopularniejszy z nich i zostanie on wykorzystany
w implementacji.

\subsection{Rust}


\subsection{Yew}


\subsection{MongoDB}


\subsection{Docker}

\section{Implementacja systemu}
\subsection{Opis API}
\subsection{Aplikacja serwerowa}
\subsection{Baza danych}
\subsection{Aplikacja użytkownika}

\section{Podsumowanie}

\section{Bibliografia}

\end{document}