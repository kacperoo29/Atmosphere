\chapter{Implementacja}
W tym rozdziale przedstawiono szczegóły implementacji systemu. 
Poszczególne zagadnienia oraz aplikacje wchodzące w jego skład
zostały opisane w oddzielnych sekcjach tego rozdziału.

\section{Komunikacja między warstwami}
Komunikacja między warstwami aplikacji odbywa się z wykorzystaniem protokołu 
HTTP (ang. \textit{Hypertext Transfer Protocol}).
Jest to jedna z technologii wykorzystywana w modelu TCP/IP 
nazywanego również protokołem Internetowym (ang. \textit{Internet Protocol}).
Jest protokołem bezstanowym \cite{http:rfc9110} służącym do przesyłania
informacji w postaci hipertekstowej i jest podstawowym rozwiązaniem
stosowanym w sieci WWW (ang. \textit{World Wide Web}), będąca systemem 
służącym do udostępniania informacji poprzez sieć Internet
z wykorzystaniem adresów URI (ang. \textit{Uniform Resource Identifier}) \cite{Jacobs:04:AWW}.

\section{Aplikacja serwerowa}
Aplikacja serwerowa została zaimplementowana jako centralny komponent systemu.
Zajmuje się zapisem odczytów w bazie danych, przetwarzaniem zapytań
pozostałych warstw, klienckiej oraz urządzeń pomiarowych,
poprzez wykorzystanie różnych metod HTTP.
Metoda POST zajmuje używana jest w nim do przekazywania danych
dla serwera w procesach takich jak tworzenie konta użytkownika,
wysyłanie nowego odczytu czy uwierzytelnianie.
Aplikacja następnie przetwarza te zapytanie oraz w odpowiedni 
sposób na nie reaguje, np. zapisując je na stałe do bazy danych 
lub wysyłając notyfikację użytkownikowi, po czym odpowiada
odpowiednim statusem klientowi.
Kolejną szeroko zastosowaną metodą jest GET. Służy ona do pobierania
zasobów z serwera. Klient wykorzystuje odpowiedni adres URI z
podstawowymi danymi lub bez, pod który wykonuje zapytanie, a
następnie serwer zwraca odpowiedź z zażądanymi informacjami.
Metodami służącymi do modyfikacji zasobów są PUT oraz DELETE.
Pierwsza z nich wykorzystywana jest do zmieniania istniejących
już zasobów lub tworzenia ich na nowo. DELETE natomiast służy
do usuwania zasobów.

Kolejnym zadaniem tej aplikacji jest również przechowywanie oraz synchronizacja konfiguracji
między pozostałymi urządzeniami w systemie. Użytkownik może ustawić podstawowe
parametry aplikacji takie jak graniczne wartości odczytów lub
częstotliwość pobierania odczytów z sensorów po czym te ustawienia
synchronizowane są z pozostałymi urządzeniami systemu.

Całość została wykonana z użyciem technologii ASP.NET Core.
Ułatwiło to znacznie implementację ze względu na bogate zasoby
tego środowiska ułatwiające pracę z protokołem HTTP oraz mnogość
bibliotek oraz wcześniej wykonanych projektów i przykładów z użyciem
tej technologii. 
Pierwszym z takich dodatkowych rozwiązań jest oficjalny sterownik
baz danych MongoDB dla środowiska .NET. Jego wykorzystanie 
umożliwiło na prostą interakcję między aplikacją oraz bazą
danych. Zamiast ręcznie dokonywać połączenia oraz przesyłać
komendy możliwa jest interakcja poprzez udostępniony interfejs.
Umożliwia on na wykorzystanie zaawansowanych technik takich jak
LINQ co znacznie ułatwiło tworzenie zapytań. Technologia ta
pomaga również w mapowaniu obiektów baz danych na klasy języka C\#.
W większości przypadków takie zależności tworzone są automatycznie,
jedynie w przypadku nadmiarowych lub braku wymaganych danych potrzebna
jest interakcja programisty. Przykładem z implementacji jest wymaganie
mapowania klas dziedziczących.
\begin{lstlisting}[language={[Sharp]C}]
BsonClassMap.RegisterClassMap<BaseUser>(cm =>
{
    cm.AutoMap();
    cm.AddKnownType(typeof(Device));
    cm.AddKnownType(typeof(User));
});
\end{lstlisting}
W tym odcinku kodu tworzona jest mapa automatyczna dla klasy BaseUser - podstawowej
użytkownika, a następnie wiązane są z nią dwa typy pochodne - Device, czyli urządzenie
oraz User, czyli użytkownik. Pozwala to na dostęp do obu typów użytkownika
z jednego kontekstu bazy danych.

Wykorzystanie ASP.NET ułatwiło również wykonanie systemu autoryzacji. 
Pakiet Microsoft.AspNetCore.Authentication zawiera często wykorzystywane metody
uwierzytelniania użytkowników. W implementacji aplikacji użyto metody tokenów JWT
(ang. \textit{JSON Web Token}). Jest to krótka, zakodowana w systemie Base64, 
reprezentacja uprawnień użytkownika używana w środowiskach z ograniczoną przestrzenią 
takich jak HTTP\cite{jwt:rfc7519}. Oprócz tego zawiera informacje o
czasie jego przedawnienia, audiencji oraz emitenta. Dzięki temu
może zostać wykorzystany w systemach pojedynczego logowania
SSO (ang. \textit{Single sign-on}) i wykorzystywany poprzez wiele aplikacji.
W implementacji został wykorzystany algorytm HmacSha256 do cyfrowego
podpisywania ważności tokenu. Sterowanie innymi parametrami takimi jak czas
przedawnienia, audiencja orz emitent odbywa się za pomocą edytowalnych
plików konfiguracyjnych.
\begin{lstlisting}[language={[Sharp]C}]
var token = new JwtSecurityToken(
  config["Issuer"],
  config["Audience"],
  claims,
  expires: DateTime.Now.Add(expiration),
  signingCredentials: new SigningCredentials(
    new SymmetricSecurityKey(Encoding.UTF8.GetBytes(config["SecretKey"])),
  SecurityAlgorithms.HmacSha256));
\end{lstlisting} 
Hasła użytkowników przechowywane są po wcześniejszym zakodowaniu algorytmem SHA256
z dodatkowymi informacjami nazywanymi solą. Są to losowe dane, które pozwalają na
dodatkowe zabezpieczenie kluczy przed wykryciem duplikatów lub ataków słownikowych
\cite{anderson2020security}.
Aby odpowiednio przejść proces uwieżytleniania użytkownik musi przesłać
swoją nazwę użytkownika oraz hasło zapytaniem z metodą POST pod odpowiedni adres
URI. Następnie jeżeli dane są prawidłowe w odpowiedzi otrzyma on wartość
token, która następnie może wykorzystać do dostępu do zablokowanych anonimowym
użytkownikom zasobów, takich jak tworzenie nowych odczytów. Wartość
ta musi zostać przesłana w nagłówku Authorization.
Użytkownicy dzielą się na różne role: administrator, użytkownik oraz urządzenie.
Administrator ma dostęp do wszystkich opcji aplikacji - konfiguracji, zarządzania odczytami
oraz użytkownikami, przeglądanie zasobów. Urządzenie może zarządzać jedynie swoimi
danymi, a zwykły użytkownik nie ma dostępu do edycji zasobów i może jedynie przeglądać
wartości.

Jako szkielet logiki aplikacji został wykorzystany wzorzec CQRS 
(ang. \textit{Command Query Responsibility Separation}).
W tym modelu akcje dzielą się na dwa typy komendy oraz zapytania \cite{fowler:cqrs}.
Komendy służą do interakcji z systemem, np. tworzenie użytkowników.
Mogą zawierać dodatkową logikę jak weryfikacja danych wejściowych 
czy sprawdzanie duplikatów.
Zapytania spełniają funkcję dostępu do danych. Pobierane oraz
filtrowane są w nich informacje z systemu i w razie potrzeby
odpowiednio transformowane.
Dzięki stosowaniu się do zasad tej metodyki aplikacja wolna
jest od efektów ubocznych i tworzenie zapytań nie zmienia
wartości systemu. Pozwala również na uproszczenie logiki
aplikacji ze względu na podział na proste moduły oraz bardzo
dobrą skalowalność z rozwiązaniami wielowątkowymi.
W implementacji tej metodyki wykorzystano bibliotekę MediatR.
Udostępnia ona bardzo łatwy w użyciu wzorzec mediatora.
Pozwala on na redukcję zależności między komponentami poprzez
zastosowanie centralnego obiektu zajmującego się komunikacją między nimi\cite{freeman2004head}.
W tym przypadku komponentami są komendy oraz zapytania i do ich realizacji
została wykorzystana dynamiczna implementacja centralnego
mediatora z biblioteki MediatR. Oddziela to dane od implementacji
zachowań i pozwala na ich niezależną modyfikacje.

Serwis notyfikacji został zaimplementowany z użyciem wzorca dekorator.
Jest to wzorzec projektowy pozwalający na dynamiczne dodawanie nowego
zachowania obiektowi bez modyfikacji innych obiektów tej samej klasy
\cite{freeman2004head}. Tym samym ułatwia on dostosowanie się do zasady
pojedynczej odpowiedzialności (ang. \textit{Single responsibility principle})
poprzez wydzielenie logiki na klasy zajmujące się tylko jednym zadaniem.
Jego wykorzystanie oznacza również zastosowanie zasady otwarte-zamknięte, w
której części systemu powinny być otwarte na rozszerzanie, ale zamknięte
na modyfikacje \cite{meyer1988object}. Jest to możliwe, gdyż dekorator pozwala
na dodanie nowych funkcjonalności obiektowi bez jego ponownego tworzenia.
Z jego wykorzystaniem użytkownik jest w stanie skonfigurować system notyfikacji,
aby ten wysyłał powiadomienia do różnych celów takich jak poczta elektroniczna
czy powiadomienia SMS.

Aby ułatwić implementację pozostałych aplikacji oraz w celu zapewnienia dokumentacji
wykorzystano bibliotekę Swashbuckle. Dostarcza ona składniki takie jak 
\begin{itemize}
  \item generator schematów OpenAPI, które mogą być następnie wykorzystane do generowania klientów
wykorzystujących dane API,
  \item generator dokumentacji bazujący na komentarzach znajdujących się w kodzie aplikacji,
  \item interfejs swagger-ui umożliwiający na przeglądanie zasobów z poziomu przeglądarki internetowej.
\end{itemize}
Kod wygenerowany na podstawie schematów wygenerowanych przez tą bibliotekę został następnie
wykorzystany zarówno w aplikacji klienckiej jak i urządzeniach pomiarowych.
Wykorzystano do tego oprogramowanie openapi-generator wspierające wiele języków
programowania oraz różnych bibliotek służących do komunikacji z użyciem protokołu HTTP.
Jest ono w stanie, na podstawie schematów OpenAPI, utworzyć klienta wraz z odpowiednimi
modelami jak również trzon aplikacji serwerowej oraz dokumentację. Pozwala również
na modyfikowanie wzorów generowanego kodu poprzez pliku mustache, za pomocą których
definiowana jest struktura wygenerowanych plików.


\section{Aplikacja kliencka}


\section{Urządzenie pomiarowe}
Do wykonywania pomiarów wykorzystany został mikrokontroler Raspberry Pi Pico W.
W celu uproszczenia tworzenia oprogramowania użyto nieoficjalnej implementacji
środowiska Arduino dla płytek z rodziny RP2040. Udostępnia ona prosty
w obsłudze poziom abstrakcji, który umożliwia na wykorzystanie głównych
możliwości mikrokontrolera takich jak odczyty z portów czy możliwość połączenia
z siecią bezprzewodową Wi-Fi. Możliwe jest dzięki temu również korzystanie z
wielu bibliotek stworzonych przez społeczność, które znacznie ułatwiają implementację.
Przykładem takiego modułu jest OneWire - biblioteka dodająca obsługę sensorów 
wykorzystujących jedną linię danych jako magistralę.
Jej wykorzystanie umożliwiło wykorzystanie jednej linii danych do połączenia wielu 
sensorów tej samej klasy.

W celu ułatwienia komunikacji z serwerem wykorzystana została zmodyfikowana wersja
generatora cpp-tiny znajdująca się w oprogramowaniu openapi-generator wykorzystując
wygenerowaną dokumentację OpenAPI 3.0. 
Ze względu na różnice w implementacji frameworku Arduino, aby uzyskać poprawnie 
wygenerowany kod zostały dokonane zmiany do plików mustache odpowiadających za składnię
generowanego kodu. W pliku Service.h.mustache, będącej nagłówkiem klasy odpowiadającej za komunikację z serwerem
oraz odpowiadającym serwisem, w celu usprawnionej obsługi nagłówków
HTTP oraz ich przechowywaniu do wykonania zapytania dodano zmienną przechowujące te nagłówki.
Podobnie, aby dostosować się do zmienionej logiki w pliku Service.cpp.mustache zmodyfikowane
zostało zapisywanie nagłówków - zamiast przechowywać je w kliencie HTTP, gdzie byłyby usuwane podczas
rozpoczęcia zapytania, przechowywane są we wcześniej stworzonej zmiennej, a następnie są one
wstrzykiwane klientowi zaraz po rozpoczęciu zapytania oraz przed jego wysłaniem.
Podobnie w przypadku pustych odpowiedzi wygenerowany kod nie był ich w stanie obsłużyć
co kończyło się dostępem do niezmapowanej pamięci. Aby to rozwiązać w funkcji odpowiedzialnej 
za pobieranie odpowiedzi dodano dodatkowe zabezpieczenie w postaci sprawdzenia
długości odpowiedzi, jeżeli jest ona mniejsza niż 0 zwracany jest pusty ciąg znaków.


