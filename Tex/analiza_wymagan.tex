\chapter{Analiza problemu}

Problem pomiarów parametrów środowiska jest powszechny w każdej dziedzinie życia.
Urządzenia pozwalające na wykonywanie prostych pomiarów takich jak temperatura
powstawały już od XVII wieku naszej ery \cite{bigotti:thermomethers}.
Są to jednak proste urządzenia niepozwalające na automatyzację oraz serializację
danych, co dzięki rozwojowi sensorów oraz systemów komputerowych jest teraz
możliwe. Wykorzystanie nowoczesnych rozwiązań i stworzenie systemu pomiarowego
może pomóc w rozwiązaniu problemów, które wcześniej były niemożliwe do rozwiązania
lub były znacznie utrudnione przez liczne czynniki. 


Zwracając uwagę na powszechność problemu oraz ilość możliwych parametrów,
które można zmierzyć, stworzony
system powinien umożliwiać na obsługę jak największej grupy rodzajów
odczytów. Możliwe jest stworzenie generycznego rozwiązania nielimitującego
użytkownika do zastosowanych sensorów oraz zbieranych czynników środowiskowych.
Może być to osiągnięte z użyciem odpowiednich poziomów abstrakcji w oprogramowaniu.
Umożliwiłoby to również na zwiększoną modularność rozwiązania oraz
wyłączenie niewymaganych w danym zastosowaniu składników systemu.
