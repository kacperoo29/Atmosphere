\chapter{Analiza problemu}

Problem pomiarów parametrów środowiska jest powszechny w każdej dziedzinie życia.
Wykonywanie ich pomaga nam w życiu codziennym przy prostych czynnościach jak
wybieranie ubioru odpowiedniego do pogody, ale również przydaje się w bardziej
zaawansowanych sferach jak badania naukowe. Częstym zastosowaniem są również
domowe hodowle roślin, które wymagają stałego monitorowania parametrów
takich jak temperatura czy wilgotność, aby zapewnić im odpowiednie warunki
do rozwoju. Innym przykładem użyteczności takiego rozwiązania są automatyczne
systemy kontroli jakości powietrza w pomieszczeniach. Mogą się one składać
z uzdatniaczy powietrza, klimatyzatorów, pieców, itp. Taki system monitorowania
może wydawać komendy tym urządzeniom w celu automatycznej i precyzyjnej kontroli.


Zwracając uwagę na powszechność problemu oraz ilość możliwych parametrów,
które można zmierzyć, stworzony system powinien umożliwiać na obsługę jak 
największej grupy rodzajów odczytów. Zależnie od przeznaczenia użytkownik
powinien mieć możliwość wykorzystania jak największej grupy sensorów ze
zdolnością do wykonywania odczytów różnych parametrów środowiska.
Taka implementacja możliwa jest poprzez stworzenie generycznego rozwiązania, które
nie ograniczałoby użytkownika i na jego podstawie tworzenie oprogramowania
sterującego danym modelem urządzenia.
Może być to osiągnięte z użyciem odpowiednich poziomów abstrakcji w oprogramowaniu.
Umożliwiłoby to również na zwiększoną modularność rozwiązania oraz
wyłączenie niewymaganych w danym zastosowaniu składników systemu.
Dzięki temu zmniejszona zostaje złożoność oraz rozmiar oprogramowania
na urządzeniu monitorującym.


Niektóre sytuacje mogą wymagać reakcji człowieka na zdarzenie, np.
w przypadku niespodziewanego uszkodzenia, którejś z części kontrolującej
temperaturę powietrza. W takich przypadkach dobrym rozwiązaniem jest możliwość
wysyłania powiadomień użytkownikowi o niestandardowym zachowaniu.
Mogą być to informacje o przekroczeniu skonfigurowanej wartości danego czynnika
lub niespodziewanej nagłej jego zmianie. Taki system notyfikacji wraz z
częstym odczytem parametrów może znacznie przyśpieszyć czas reakcji
i naprawy problemu, co może przełożyć się na znacznie zredukowane straty.
Taki system mógłby zostać również wykorzystany do notyfikowania innych
urządzeń o zmianach parametrów tym samym pozwalając na automatyzację
ich działania oraz konfiguracji. Urządzenie mogłoby reagować na daną 
informację i sama dostosować swoje działanie lub w bardziej zaawansowanych
przypadkach logika biznesowa mogłaby znaleźć się po stronie serwera i 
kontrolować pracę tego urządzenia.


Do wielu zastosowań może być również konieczne przeglądanie odczytów historycznych.
Taka funkcjonalność jest konieczna w przypadku, np. badań wpływu danych
parametrów środowiska na badany obiekt. Umożliwia to na ułatwioną korelację
wartości odczytów ze zmianami w podmiocie badań. Przydaje się również 
wraz z wykorzystaniem systemów poprawiających jakość czy temperaturę
powietrza, gdzie porównując wykorzystaną moc urządzenia możemy skorelować
ze zmianami parametrów środowiska wraz z biegiem czasu.
Aplikacja powinna więc mieć możliwość zapisu oraz odczytu danych. Powinny być
one przechowywane w formacie, który umożliwi późniejsze ich łatwe przetwarzanie,
co pomoże w przypadku budowy innych funkcjonalności na zgromadzonych danych
jak również ułatwi samą ich prezentację użytkownikowi. Funkcjonalnością, która
powinna się również pojawić powinna być możliwość filtrowania oraz sortowania
odczytów przez użytkownika.
