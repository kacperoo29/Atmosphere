\chapter{Podsumowanie}
Wytworzona aplikacja spełnia pierwotne założenia pracy. Umożliwia ona na 
wykorzystanie funkcjonalności tj. zbieranie oraz odczyt parametrów środowiska,
przeglądanie danych historycznych oraz powiadamianie użytkownika o niespodziewanych
sytuacjach. Dodatkowo system posiada możliwość uwierzytelniania oraz autoryzacji 
użytkowników oraz urządzeń w celu zapewnienia bezpieczeństwa przechowywanych danych.
Dzięki zastosowanym wzorcom projektowym oraz modularnej architekturze aplikacja
jest podatna na rozwój w przyszłości o nowe funkcjonalności oraz rozszerzenie
aktualnie istniejących rozwiązań. Możliwym wektorem dalszego rozwoju jest
uproszczenie interfejsu oraz procesu wprowadzania w celu zwiększenia dostępności
dla większej audiencji. Aktualna implementacja obiera za cel użytkowników zaznajomionych
z systemami komputerowymi oraz wymaga od nich dodatkowej wiedzy, np. przy konfiguracji
urządzeń pomiarowych. 

Ze względu na obszerność problemu trudnym jest wytworzenie systemu odpowiadającego
każdej sytuacji, jednak powtarzająca się funkcjonalność umożliwiła wytworzenie
solidnej bazy oprogramowania, które może zostać rozwinięte do szczególnych zastosowań.
