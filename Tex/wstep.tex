\chapter{Wstęp}

Parametry środowiska są ważnym współczynnikiem zarówno codziennego życia jak
i dziedzin nauki oraz przemysłu. Najtrafniejszym przykładem jest temperatura, 
która wpływa w znaczny sposób na działania człowieka, będąc również istotnym
parametrem w agrokulturze oraz innych branżach produkcyjnych. Z powodu powszechności
tego problemu zasadnym jest stworzenie urządzeń i systemów ułatwiających badanie
oraz kontrolę tych parametrów. Często również wykorzystywane są one w różnego rodzaju
analizach przez co dobrym pomysłem jest umożliwienie przechowywania odczytanych
wcześniej danych.


Celem pracy jest stworzenie systemu pozwalającego na odczyt parametrów środowiska
w pomieszczeniach oraz ich przechowywanie. Stworzone oprogramowanie powinno pozwalać
na zapis dowolnych danych z różnego rodzaju sensorów oraz ich późniejszy odczyt
przez użytkownika. Część systemu zajmująca się monitorowaniem oraz przesyłem
parametrów została zrealizowana przy użyciu mikrokontrolera Raspberry Pi, a
interfejs użytkownika przyjęła formę strony internetowej. 

W następnych rozdziałach zostały opisane następujące zagadnienia:
\begin{itemize}
  \item podobne rozwiązania wraz z ich wadami oraz zaletami,
  \item wybrane technologie do implementacji systemu,
  \item szczegóły implementacji systemu.
\end{itemize}
