\chapter{Wstęp}

Problem pomiarów parametrów środowiska jest powszechny w każdej dziedzinie życia.
Wykonywanie ich pomaga nam w życiu codziennym przy prostych czynnościach jak
wybieranie ubioru odpowiedniego do pogody, ale również przydaje się w bardziej
zaawansowanych sferach jak badania naukowe. Częstym zastosowaniem są również
domowe hodowle roślin, które wymagają stałego monitorowania parametrów
takich jak temperatura czy wilgotność, aby zapewnić im odpowiednie warunki
do rozwoju. Innym przykładem użyteczności takiego rozwiązania są automatyczne
systemy kontroli jakości powietrza w pomieszczeniach. Mogą się one składać
z uzdatniaczy powietrza, klimatyzatorów, pieców, itp. Taki system monitorowania
może wydawać komendy tym urządzeniom w celu automatycznej i precyzyjnej kontroli
w czasie zbliżonym do rzeczywistego.

Celem pracy jest stworzenie systemu pozwalającego na odczyt parametrów środowiska
w pomieszczeniach oraz ich przechowywanie. Stworzone oprogramowanie powinno pozwalać
na zapis dowolnych danych z różnego rodzaju sensorów oraz ich późniejszy odczyt
przez użytkownika. Powinien również powstać system powiadomień informujący
użytkownika o niespodziewanych wartościach odczytów.
Część systemu zajmująca się monitorowaniem oraz przesyłem
parametrów została zrealizowana przy użyciu komputera Raspberry Pi, a
interfejs użytkownika przyjął formę strony internetowej. Całość komunikuje się
z centralnym serwerem, który przechowuje oraz udostępnia dostęp do
zapisanych na nim danych.

W następnych rozdziałach zostały opisane następujące zagadnienia:
\begin{itemize}
  \item podobne rozwiązania wraz z ich wadami oraz zaletami. 
    W tym rozdziale przedstawiono podobne technologie do wykonywanego
    systemu oraz porównano je z wymaganiami postawionymi implementowanemu
    systemowi.
  \item szczegółowa analiza problemu monitorowania parametrów środowiska.
    Opisano tam zastosowania takiego systemu oraz wymagania jakie powinien
    on spełniać. Uwzględniono wymagania funkcjonalne oraz niefunkcjonalne, 
    do których zastosowano się podczas tworzenia systemu.
  \item wybrane technologie do implementacji systemu. Rozdział
    zawiera krótkie opisy wybranych technologii oraz czynniki jakie
    wpłynęły na ich wybór w celu implementacji.
  \item szczegóły implementacji systemu. Rozdział zawiera opisy
    architektury oraz implementacji wybranych funkcjonalności
    systemu. Podzielony został na sekcje odwołujące się do trzech
    komponentów: serwera, aplikacji klienckiej oraz urządzenia pomiarowego.
  \item opis wykonanych testów. Zawiera informacje o przeprowadzonych
    testach jednostkowych oraz funkcjonalnych.
  \item prezentacja systemu. Zaprezentowano tam prototyp urządzenia
    monitorującego parametry środowiska jak i również aplikacji klienckiej.
  \item podsumowanie wraz z oceną zaimplementowanego rozwiązania. Opisano
    w jakim stopniu zrealizowane zostało rozwiązanie postawionego problemu
    oraz możliwe jego usprawnienia.
\end{itemize}
