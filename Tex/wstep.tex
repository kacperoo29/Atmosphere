\chapter{Wstęp}

Parametry środowiska są ważnym współczynnikiem zarówno codziennego życia jak
i dziedzin nauki oraz przemysłu. Najtrafniejszym przykładem jest temperatura, 
która wpływa w znaczny sposób na działania człowieka, będąc również istotnym
parametrem w agrokulturze oraz innych branżach produkcyjnych. Oddziałują
one również w życiu każdego człowieka na jego samopoczucie oraz zachowanie.
Podobnie jest w różnych dziedzinach nauki, gdzie do badań używane są odczyty
danych parametrów środowiska do budowania modeli czy sprawdzania zależności
między nimi a badanym zjawiskiem. Z powodu powszechności
tego problemu zasadnym jest stworzenie urządzeń i systemów ułatwiających badanie
oraz kontrolę tych parametrów. Tym samym nie powinien on ograniczać użytkownika
co do badanych parametrów oraz ich przeznaczeniu.

Celem pracy jest stworzenie systemu pozwalającego na odczyt parametrów środowiska
w pomieszczeniach oraz ich przechowywanie. Stworzone oprogramowanie powinno pozwalać
na zapis dowolnych danych z różnego rodzaju sensorów oraz ich późniejszy odczyt
przez użytkownika. Powinien również powstać system powiadomień informujący
użytkownika o niespodziewanych wartościach odczytów.
Część systemu zajmująca się monitorowaniem oraz przesyłem
parametrów została zrealizowana przy użyciu mikrokontrolera Raspberry Pi, a
interfejs użytkownika przyjął formę strony internetowej. Całość komunikuje się
z centralnym serwerem, który przechowuje oraz udostępnia dostęp do
zapisanych na nim danych.

W następnych rozdziałach zostały opisane następujące zagadnienia:
\begin{itemize}
  \item podobne rozwiązania wraz z ich wadami oraz zaletami,
  \item wybrane technologie do implementacji systemu,
  \item szczegóły implementacji systemu.
\end{itemize}
